\documentclass[11pt]{article}
\usepackage{times}
\usepackage{amsmath,amsthm,amssymb,mathtools,setspace,enumitem,epsfig,titlesec,verbatim,color,array,multirow,comment,graphicx,hyperref,blkarray}
%\usepackage[sort&compress]{natbib} % ProcB
\usepackage[super,sort&compress,comma]{natbib} % NComms
\usepackage[bf,small]{caption}
\usepackage[export]{adjustbox}
\usepackage[top=2.5cm,left=2.8cm,right=2.8cm,bottom=3.2cm]{geometry} 
\smallskip 

\definecolor{darkred}{rgb}{0.6,0,0}
\definecolor{darkblue}{rgb}{0,0.3,0.8}

\newcommand{\christian}[1]{\textcolor{blue}{{\bf CH:} #1}} 

\titleformat{\section}{\sffamily \fontsize{12}{20}\bfseries}{\thesection}{1em}{}
\titleformat{\subsection}{\sffamily \fontsize{11}{20}\bfseries}{\thesubsection}{1em}{}

\renewcommand{\figurename}{Supplementary Figure}


\newcommand{\FigIllustration}{{\bf Fig.~1}}

\newtheoremstyle{plainCl1}% name
{9pt}%      Space above, empty = 'usual value'
{15pt}%      Space below
{\it}% 	   Body font
{}%         Indent amount (empty = no indent, \parindent = para indent)
{\bfseries}% Thm head font
{.}%        Punctuation after thm head
{2mm}% Space after thm head: \newline = linebreak
{}%         Thm head spec

\newtheoremstyle{plainCl2}% name
{9pt}%      Space above, empty = 'usual value'
{15pt}%      Space below
{\it}% 	   Body font
{}%         Indent amount (empty = no indent, \parindent = para indent)
{\bfseries}% Thm head font
{.}%        Punctuation after thm head
{4mm}% Space after thm head: \newline = linebreak
{}%         Thm head spec

\theoremstyle{plainCl1}
\newtheorem{theorem}{Theorem}
\newtheorem{Prop}{Proposition}
\newtheorem{definition}{Definition}

\theoremstyle{plainCl2}
\newtheorem{lemma}{Lemma}
\newtheorem{proposition}{Proposition}
\newtheorem{Corollary}{Corollary}


\newcommand{\ALLD}{\emph{D}}

\newcommand{\A}{\mathbf{A}}
\newcommand{\abf}{\mathbf{a}}
\newcommand{\T}{\mathbf{T}}
\newcommand{\ubf}{\mathbf{u}}
\newcommand{\C}{\mathrm{C}}
\newcommand{\D}{\mathrm{D}}

\title{\sffamily \Large Supplementary Information\\[0.1cm] {\bfseries Introspection dynamics in asymmetric multiplayers games}}
\date{\empty}
\author{\parbox[c]{16cm}{\centering \onehalfspacing \fontsize{11}{12}\selectfont Marta Couto$^1$ and Saptarshi Pal$^1$\\[0.2cm]
$^1$Max Planck Research Group Dynamics of Social Behavior, Max Planck Institute for Evolutionary Biology, 24306~Ploen, Germany}}


\begin{document}
\maketitle
\onehalfspacing
\section*{Abstract}
\newpage
\section*{Introduction}
\section*{Model}
We consider the normal form game with $N$ players where $N > 2$. In the game, player $i$ can play an action from their action set, $\A^i := \{a^i_1, a^i_2, ..., a^i_{m^i}\}$. The action set of player $i$ has $m^i$ actions. There are, therefore, $m^1 \times m^2 \times ... \times m^N$ distinct states of the game. We denote a state of the game with $\abf$ where $\abf := (\abf^1, \abf^2, ..., \abf^N) \in \A^1 \times \A^2 \times ... \times \A^N$. We also use the notation, $\abf := (\abf^i, \abf^{-i})$ to denote the state from the perspective of player $i$. In the state $\abf$, player $i$ plays the action $\abf^i \in \A^i$ and their co-players play the action $\abf^{-i} \in \prod_{j \neq i} \A^j$.
\\ \\
\noindent The payoffs of a player depends on the state of the game. We denote the payoff of player $i$ in the state $\abf$ with $\pi^i(\abf)$.

 
\newpage
\section*{Appendix: Proofs}
\label{Section:Appendix}
\begin{proof}
\textbf{Proof of Proposition} \ref{Th:additive-games-stationary-dist} \\ \\ 
Since $\beta$ is finite, the transition matrix of the process $\T$ given by Eq. \ref{Eq:transition-matrix} is primitive and therefore, the stationary distribution $\ubf = (\ubf_\abf)_{\abf \in \A}$ of the row-stochastic transition matrix is unique and satisfies the conditions laid out in Eq. \ref{Eq:lefteigenvector} and \ref{Eq:normalizationcondition}. To continue for the rest of the proof, we introduce some short-cut notation that will be of use later in the proof:\\
\begin{align}
I_{q} &:= I(\abf_q,\abf), \quad \mathit{iff} \quad \abf_q \in \mathit{Neb}(\abf) \\ \notag \\ 
\label{Eq:shortcut-tau}
\tau^j_{\abf^j} &:= \frac{1}{\displaystyle \sum_{a' \in \A^j} e^{\beta \left( \pi^j_{a'} -  \pi^j_{\abf^j} \right) }} 
\end{align}
\noindent In order to show that the candidate stationary distribution, as proposed in Eq. \ref{Eq:additive-game-stationary-distribution} is the stationary distribution of the process, we need to show that the following are true:
\begin{eqnarray}
\label{step-one}
\T_{\abf,\abf} \ubf_\abf  + \sum_{\abf_q \neq \abf} \T_{\abf_q, \abf} \ubf_{\abf_q}= \ubf_a \quad \forall \abf \in \A \\ 
\label{step-two}
\sum_{\abf \in \A} \ubf_\abf  = 1
\end{eqnarray}
The Eq \ref{step-one} can be simplified further with the steps: 
\begin{align}
&\T_{\abf,\abf} \ubf_\abf  + \sum_{\abf_q \neq \abf} \T_{\abf_q, \abf} \ubf_{\abf_q}= \\
& \left( 1 - \frac{1}{N} \sum_{\abf_q \in \mathit{Neb}(\abf)} \frac{1}{m^{I_q}-1} \cdot p^{I_q}_{\abf^{I_q} \to \abf^{I_q}_q} \cdot \ubf_{\abf} \right) + \sum_{\abf_q \in \mathit{Neb}(\abf)}  \frac{1}{m^{I_q}-1} \cdot p^{I_q}_{\abf^{I_q}_q \to \abf^{I_q}} \cdot \ubf_{\abf_q} = \\ \notag \\
\label{eq:important-step}
& \ubf_\abf +  \frac{1}{N} \sum_{\abf_q \in \mathit{Neb}(\abf)} \left( \prod_{k \neq I_q} \tau^k_{\abf^k} \right) \left( p^{I_q}_{\abf^{I_q}_q \to \abf^{I_q}} \cdot \tau^{I_q}_{\abf^{I_q}} -  p^{I_q}_{\abf^{I_q} \to \abf^{I_q}_q} \cdot \tau^{I_q}_{\abf^{I_q}_q} \right) \cdot \left(  \frac{1}{m^{I_q}-1} \right)
\end{align}
\\ Using the definition of probability of update of the introspection dynamics, as given by Eq. \ref{Eq:introspection-update} and Eq. \ref{Eq:shortcut-tau}, it can be shown that: 
\begin{equation}
p^{I_q}_{\abf^{I_q}_q \to \abf^{I_q}} \cdot \tau^{I_q}_{\abf^{I_q}} -  p^{I_q}_{\abf^{I_q} \to \abf^{I_q}_q} \cdot \tau^{I_q}_{\abf^{I_q}_q} = 0
\label{important-step-is-zero}
\end{equation}
\\ \noindent Plugging the equality in Eq. \ref{important-step-is-zero} into Eq. \ref{eq:important-step}, we can see that the left hand side of Eq. \ref{step-one} indeed simplifies to $\ubf_{\abf}$. Now, to confirm that the candidate $\ubf$ is the unique stationary distribution we need to check if Eq. \ref{step-two} holds. Simplifying the left hand side of this equation shows that:
\begin{align}
\sum_{\abf \in \A} \ubf_\abf &= \sum_{\abf \in \A} \prod_{k=1}^N \tau^k_{\abf^k} \\ \notag \\
\label{step-prod-sum-sum-prod}
&= \left( \prod_{k=1}^N \sum_{\abf' \in \A} \displaystyle e^{\beta \pi^k_{\abf'}}\right)^{-1} \cdot \left( \sum_{\abf \in \A} \prod_{k=1}^N \displaystyle e^{\beta \pi^k_{\abf^k}} \right) \\ \notag \\
\label{step-prod-is-one}
&= 1
\end{align}
\noindent The step from Eq. \ref{step-prod-sum-sum-prod} to Eq. \ref{step-prod-is-one} is possible because the sum and product in Eq. \ref{step-prod-sum-sum-prod} are interchangeable for both the terms. Therefore, condition Eq. \ref{step-two} is satisfied too. 
\end{proof}

\begin{proof}
\textbf{Proof of Proposition} \ref{Th:additive-game-product-of-marginals} \\ \\ 
If $\ubf$ is the unique stationary distribution of the $N-$player additive game with under the finite selection introspection dynamics, it is given by the expression in Eq. \ref{Eq:additive-game-stationary-distribution}. We calculate the marginal distribution of any arbitrary state $\abf$, $\xi_{\abf} = (\xi^j_{\abf^j})_{j = 1,2,...,N}$ by using the definition of marginal distribution in Eq. \ref{Eq:marginal-definition}. It follows that: 
\begin{align}
\xi^j_{\abf^j} &= \sum_{\mathbf{b} \in \A^{-j}} \ubf_{(\abf^j,\mathbf{b})} \\ \notag \\
&= \left( \prod_{k=1}^N \sum_{\abf' \in \A^k} \displaystyle e^{\beta \pi^k_{\abf'}}\right)^{-1} \cdot \displaystyle e^{\beta \pi^j_{\abf^j}} \cdot \left( \sum_{\mathbf{b} \in \A^{-j}} \prod_{k \neq j} e^{\beta \pi^k_{\mathbf{b}^k}} \right) \\ \notag \\
&= \left( \sum_{\abf' \in \A^j} \displaystyle e^{\beta \pi^j_{\abf'}}\right)^{-1} \cdot \displaystyle e^{\beta \pi^j_{\abf^j}} \cdot \left( \prod_{k\neq j} \sum_{\abf' \in \A^k} \displaystyle e^{\beta \pi^k_{\abf'}}\right)^{-1} \cdot \left( \sum_{\mathbf{b} \in \A^{-j}} \prod_{k \neq j} e^{\beta \pi^k_{\mathbf{b}^k}} \right) \\ \notag \\ 
&= \left( \sum_{\abf' \in \A^j} \displaystyle e^{\beta \pi^j_{\abf'}}\right)^{-1} \cdot \displaystyle e^{\beta \pi^j_{\abf^j}} \cdot \left( \sum_{\abf' \in \A^{-j}} \prod_{k\neq j}  \displaystyle e^{\beta \pi^k_{\abf'}}\right)^{-1} \cdot \left( \sum_{\mathbf{b} \in \A^{-j}} \prod_{k \neq j} e^{\beta \pi^k_{\mathbf{b}^k}} \right) \\ \notag \\ 
&= \left( \sum_{\abf' \in \A^j} \displaystyle e^{\beta \left( \pi^j_{\abf'} - \pi^j_{\abf^j}\right)}\right)^{-1}
\end{align}
\noindent Therefore, the marginal distribution follows Eq.  \ref{Eq:marginal-at-additive-game}. Now since we also additionally know that the stationary distribution follows the form Eq. \ref{Eq:additive-game-stationary-distribution}, we can conclude that for additive games, under introspection dynamics with finite selection, Eq. \ref{Eq:additive-game-products} holds. 
\end{proof}

\begin{proof}
\textbf{Proof of Proposition} \ref{prop:stationary-dist-lpgg} \\ \\
Since we have demonstrated that the linear public goods game is an additive game, the proof of this theorem can be performed by directly using Theorem \ref{Th:additive-games-stationary-dist}. Here, we provide an independent proof. The idea behind this proof is identical to the proof of Theorem \ref{Th:additive-games-stationary-dist}. \\ \\
\noindent The stationary transition matrix $\T$ for the linear public goods game is primitive when $\beta$ is finite (i.e., there is a positive power $k$ such that $\T^k$ is a strictly positive matrix). Therefore, the stationary distribution of $\T$ will always be unique. We define the following short cut notations for the ease of the proof: 
\begin{eqnarray}
\bar{\abf}^j &:= \{\D,\C\} \setminus \abf^j  \\ 
p^j &:= \frac{1}{1 + \displaystyle e^{\beta f(c^j, r^j)}} 
\end{eqnarray}
In addition we introduce a mapping function $\alpha(.)$ which maps the action $\C$ to 1 and the action $\D$ to 0. That is $\alpha(\C) := 1$ and $\alpha(\D) := 0$. Using these notations and Eq. \ref{Eq:introspection-update} and \ref{Eq:difference-payoffs-lpgg} we can write the probability that a player $j$ updates from $\abf^j$ to $\bar{\abf}^j$ while their co-players play $\abf^{-j}$ as: \\
\begin{equation}
p^j_{\displaystyle \bar{\abf}^j  \to \abf^j} (\abf^{-j}) = p^j \mathit{sign}(\abf^j) + \alpha(\bar{\abf}^j) \\ \\
\end{equation}
The candidate stationary distribution $\ubf$ given in Eq \ref{Eq:stationary_dist_lpgg} can be written down using our shortcut notation as: 
\begin{equation}
\label{Eq:stationary-dist-shortcut}
\ubf_\abf = \prod_{k = 1}^{N}  p^k \mathit{sign}(\abf^k) + \alpha(\bar{\abf}^k) \quad ,\forall \abf \in \{0,1\}^N
\end{equation}
This stationary distribution must satisfy the following properties, which are also given in Eq  \ref{Eq:lefteigenvector} and \ref{Eq:normalizationcondition}:
\begin{align}
\label{Eq:transition-in-proof}
&\ubf_\abf = \T_{\abf,\abf} \ubf_\abf  + \sum_{\abf_q \neq \abf} \T_{\abf_q, \abf} \ubf_{\abf_q}  \\[10pt] 
\label{Eq:normalization-in-proof}
&\sum_{\forall \abf_q} \ubf_{\abf_q}= 1
\end{align}
Where, the terms in the right hand side of Eq. \ref{Eq:transition-in-proof} can be simplified using Eq \ref{Eq:introspection-update} and \ref{Eq:transition-matrix} as follows:
\begin{eqnarray}
\T_{\abf,\abf} = 1 - \sum_{k=1}^{N} \T_{(\abf^k, \abf^{-k}), (\bar{\abf}^k,\abf^{-k})} = 1 - \frac{1}{N} \sum_{k=1}^{N} p^k \textit{sign}(\bar{\abf}^k) + \alpha(\abf^k)
\label{Eq:first-term-rhs-proof}
\end{eqnarray} 
and additionally, also using Eq. \ref{Eq:stationary-dist-shortcut} the second term can be simplified too:
\begin{align}
\sum_{\abf_q \neq \abf} \T_{\abf_q, \abf} \ubf_{\abf_q} &= \sum_{k = 1}^N \T_{(\bar{\abf}^k,\abf^{-k}), (\abf^k, \abf^{-k})} \ubf_{(\bar{\abf}^k,\abf^{-k})} \\[10pt]
&= \frac{1}{N} \sum_{k = 1}^N \left(p^k \textit{sign}(\abf^k) +\alpha(\bar{\abf}^k) \right) \ubf_{(\bar{\abf}^k,\abf^{-k})} \\[10pt] 
\label{Eq:second-term-rhs-proof}
&= \frac{\ubf_\abf}{N} \sum_{k=1}^{N} p^k \textit{sign}(\bar{\abf}^k) + \alpha(\abf^k) 
\end{align}
Now, using Eq. \ref{Eq:first-term-rhs-proof}, \ref{Eq:second-term-rhs-proof} one can show that the right hand side of Eq. \ref{Eq:transition-in-proof} is the element of the stationary distribution, corresponding to the state $\abf$, $\ubf_a$.  Now, to complete the proof, we must show that Eq. \ref{Eq:normalization-in-proof} is also true for our candidate stationary distribution. This can be done by decomposing the sum of the elements of the stationary distribution as follows:
\begin{align}
\sum_{\forall \abf_q} u_{\abf_q} =& \displaystyle \sum_{\forall \abf_q} \prod_{k=1}^N p^k \mathit{sign}(\abf_q^k) + \alpha(\bar{\abf}_q^k) \\[10pt]
=& \displaystyle \displaystyle \sum_{\forall \abf_q} (1-p^N)  \prod_{k=1}^{N-1} p^k \mathit{sign}(\abf_q^k) + \alpha(\bar{\abf}_q^k)  + p^N  \prod_{k=1}^{N-1} p^k \mathit{sign}(\abf_q^k) + \alpha(\bar{\abf}_q^k) \\[10pt]
=& \displaystyle \sum_{\forall \abf_q} \prod_{k=1}^{N-1} p^k \mathit{sign}(\abf_q^k) + \alpha(\bar{\abf}_q^k)
\end{align}
When the above decomposition is perfomed $N-1$ more times, the sum of the right hand side becomes 1. This prooves that the candidate stationary distribution is also a probability distribution.
\end{proof}

\begin{proof}
\textbf{Proof of Proposition} \ref{Prop:Symmetric-2-strategies-state} \\ \\
By construction, the candidate stationary distribution given by Eq. \ref{Eq:stationary-dist-symm-2-stgs-state} and Eq. \ref{Eq:stationary-dist-normalization-symm-2-stgs-state} is a probability distribution since it satisfies the condition in Eq. \ref{Eq:normalizationcondition} and for any state $\abf'$, $\ubf_{\abf'}$ is between 0 and 1.  Moreover, since $\beta$ is finite, the transition matrix of the process $\T$ is primitive and therefore, it will have a unique stationary distribution. To show that the candidate stationary distribution is the unique stationary distribution, we need to check if for this process, $\ubf \T= \ubf$. That is, the condition in Eq. \ref{Eq:transition-in-proof} must hold for all states $\abf$. We re-introduce some notations that we will use in this proof: 
\begin{align}
\bar{\abf}^j &:= \{\D,\C\} \setminus \abf^j \\[10pt]
\alpha(a)&:= 
\begin{cases}
1 \quad \text{if} \quad a = \C \\[10pt]
0 \quad \text{if} \quad a = \D 
\end{cases}\\[10pt]
\mathcal{C}(\abf) &= \sum_{j=1}^N \alpha(\abf^j)
\end{align}
For this process, the first term in the right hand side of Eq. \ref{Eq:transition-in-proof} can be simplified as: 
\begin{align}
\ubf_{\abf} \T_{\abf,\abf}  &= \ubf_\abf - \ubf_{\abf} \sum_{k=1}^{N} \T_{(\abf^k, \abf^{-k}),(\bar{\abf}^{k}, \abf^{-k})} \\[10pt]
&= \ubf_{\abf} - \frac{\ubf_{\abf}}{N} \sum_{k=1}^N \frac{1}{1 + \displaystyle e^{\mathit{sign}(\bar{\abf}^{k}) \beta f(N_k)}}
\label{Eq:T_aa_u_a term}
\end{align}
\\ \noindent Where, the function $\mathit{sign}(.)$ is  defined as in Eq. \ref{Eq:sign-function} and$f(j)$ is the difference in payoffs between playing $\D$ and $\C$ when there are $j$ co-players playing $\C$. The term $N_k$ is the number of co-players of $k$ that play $\C$ in state $\abf$. That is,
\begin{equation}
N_k := \sum_{j \neq k} \alpha(\abf^j)
\end{equation}
The second term in the right hand side of Eq. \ref{Eq:transition-in-proof} can be simplified as, 
\begin{align}
\sum_{\abf_q \neq \abf} \T_{\abf_q, \abf} \ubf_{\abf_q} &= \sum_{k=1}^N \T_{(\bar{\abf}^k, \abf^{-k}),(\abf^k, \abf^{-k})} \ubf_{(\bar{\abf}^k, \abf^{-k})} \\[10pt]
\label{Eq:step-single-product}
&= \frac{1}{N \Gamma} \sum_{k=1}^N \T_{(\bar{\abf}^k, \abf^{-k}),(\abf^k, \abf^{-k})} \displaystyle \prod_{j=1}^{\mathcal{C}((\bar{\abf}^k, \abf^{-k}))} e^{-\beta f(j-1)}\\[10pt]
&=  \frac{1}{N \Gamma} \sum_{k=1}^N \T_{(\bar{\abf}^k, \abf^{-k}),(\abf^k, \abf^{-k})} \displaystyle \left( \prod_{j=1}^{N_k}  e^{-\beta f(j-1)} \right) \cdot e^{-\beta \alpha(\bar{\abf}^k)f(-\alpha(\abf^k)+ N_k}
\label{Eq:step-product-broken}
\end{align}
Between the steps, Eq. \ref{Eq:step-single-product} and \ref{Eq:step-product-broken}, we took out one term from the product that is present in our candidate distribution. This term accounts for the $k^{th}$ players action in the neighbouring state $(\bar{\abf}^k, \abf^{-k})$ of $\abf$. For simplicity, we replace $\T_{(\bar{\abf}^k, \abf^{-k}),(\abf^k, \abf^{-k})}$ with just $\T$ in the next steps. We continue the simplification of Eq. \ref{Eq:step-product-broken} in the later steps by introducing terms that cancel each other. \\[10pt]
\begin{align}
\sum_{\abf_q \neq \abf} \T_{\abf_q, \abf} \ubf_{\abf_q} &=  \frac{1}{N \Gamma} \sum_{k=1}^N \T \cdot \displaystyle \left( \prod_{j=1}^{N_k}  e^{-\beta f(j-1)} \right) \cdot \frac{e^{-\beta \alpha(\bar{\abf}^k)f(-\alpha(\abf^k)+ N_k)}}{e^{-\beta \alpha(\abf^k)f(-\alpha(\bar{\abf}^k) +N_k)}} \cdot e^{-\beta \alpha(\abf^k)f(-\alpha(\bar{\abf}^k) + N_k)}
\label{Eq:new-cancelable-term-introduced}
\end{align}
\\ \noindent The newly introduced term in Eq. \ref{Eq:new-cancelable-term-introduced} can be taken inside the product. Note that this term is 0 if the $k^{th}$ player plays $\D$ in the state $\abf$. When this term is taken inside the product bracket, products of exponent $e^{-\beta f(j-1)}$ can be performed for $j$ ranging from $1$ to the number of cooperators in state $\abf$, $\mathcal{C}(\abf)$. This product is then the stationary distribution probability $\ubf_\abf$. That is, 
\begin{align}
\sum_{\abf_q \neq \abf} \T_{\abf_q, \abf} \ubf_{\abf_q} &=  \frac{1}{N \Gamma} \sum_{k=1}^N \T \cdot \displaystyle \left( \prod_{j=1}^{N_k}  e^{-\beta f(j-1)} \cdot e^{-\beta \alpha(\abf^k)f(-\alpha(\bar{\abf}^k) + N_k)} \right) \cdot \frac{e^{-\beta \alpha(\bar{\abf}^k)f(-\alpha(\abf^k)+ N_k)}}{e^{-\beta \alpha(\abf^k)f(-\alpha(\bar{\abf}^k) +N_k)}} \\[10pt]
&= \frac{1}{N} \sum_{k=1}^N  \T \cdot \left( \frac{1}{\Gamma} \prod_{j=1}^{\mathcal{C}(\abf)} e^{-\beta f(j-1)}\right) \cdot \frac{e^{-\beta \alpha(\bar{\abf}^k)f(-\alpha(\abf^k)+ N_k)}}{e^{-\beta \alpha(\abf^k)f(-\alpha(\bar{\abf}^k) +N_k)}} \\[10pt]
&= \frac{1}{N} \sum_{k=1}^N  \T_{(\bar{\abf}^k, \abf^{-k}),(\abf^k, \abf^{-k})} \cdot \ubf_\abf \cdot \frac{e^{-\beta \alpha(\bar{\abf}^k)f(-\alpha(\abf^k)+ N_k)}}{e^{-\beta \alpha(\abf^k)f(-\alpha(\bar{\abf}^k) +N_k)}} \label{Eq: last-step-w-fraction} 
\end{align}
\\ \noindent The fraction inside the sum in  Eq. \ref{Eq: last-step-w-fraction} can be simplified as follows leading to further simplification of Eq. \ref{Eq: last-step-w-fraction}:
\begin{align}
\sum_{\abf_q \neq \abf} \T_{\abf_q, \abf} \ubf_{\abf_q} &= \frac{1}{N} \sum_{k=1}^N  \T_{(\bar{\abf}^k, \abf^{-k}),(\abf^k, \abf^{-k})} \cdot \ubf_\abf \cdot e^{sign(\abf^k) \beta f(N_k)} 
\label{Eq:only-need-to-replace-T-now}
\end{align}
\\ \noindent In Eq. \ref{Eq:only-need-to-replace-T-now} we can replace the element of the transition matrix $\T_{(\bar{\abf}^k, \abf^{-k}),(\abf^k, \abf^{-k})}$ by using the following: 
\begin{equation}
\T_{(\bar{\abf}^k, \abf^{-k}),(\abf^k, \abf^{-k})} = \frac{1}{1 + \displaystyle e^{\mathit{sign}(\abf^k) \beta f(N_k)}} 
\label{Eq:transition-matrix-T-symm-2-stgs}
\end{equation}
\\ \noindent Using the expression for the transition matrix element from Eq. \ref{Eq:transition-matrix-T-symm-2-stgs} into Eq. \ref{Eq:only-need-to-replace-T-now} and by using Eq. \ref{Eq:T_aa_u_a term}, we can simplify further: 
\begin{align}
\sum_{\abf_q \neq \abf} \T_{\abf_q, \abf} \ubf_{\abf_q} &= \frac{\ubf_\abf}{N} \sum_{k=1}^N \frac{1}{1 + \displaystyle e^{\mathit{sign}(\abf^k) \beta f(N_k)}} \cdot  e^{sign(\abf^k) \beta f(N_k)} \\[10pt]
&= \frac{\ubf_\abf}{N} \sum_{k=1}^N \frac{1}{1 + \displaystyle e^{\mathit{sign}(\bar{\abf}^k) \beta f(N_k)}}  \\[10pt]
&= \ubf_\abf - \ubf_\abf \T_{\abf,\abf}
\end{align}
\\ \noindent The final step in the previous simplification shows that Eq. \ref{Eq:transition-in-proof} holds for any $\abf \in \{\C,\D\}^N$. Therefore, the candidate distribution we propose in Eq. \ref{Eq:stationary-dist-symm-2-stgs-state} is the unique stationary distribution of the symmetric $N$-player game with two strategies. 
\end{proof}
\newpage
\begin{proof}
\textbf{Proof of Corollary} \ref{Lemma: Symmetric-2-stg} \\ \\
To show this result we count how many states are identical to a state $\abf \in \{\C,\D\}^N$ in a symmetric game. When players are symmetric in a two-strategy game, states can be enumerated by counting the number of $\C$ players in that state. This can also be confirmed by the expression of the stationary distribution in Eq. \ref{Eq:stationary-dist-symm-2-stgs-state}. Two distinct states $\abf, \abf'$ having the same number of cooperators (i.e., $\mathcal{C}(\abf') = \mathcal{C}(\abf)$), have the same stationary distribution probability (i.e., $\ubf_{\abf'} = \ubf_{\abf}$).
\\ \\ 
\noindent In a game with $N$ players, there can be $k$ players playing $\C$ in exactly $N \choose k$ ways. As argued before, all of these states are identical and are also equiprobable in the stationary distribution. Therefore, the stationary distribution probability of having $k$, $\C$ players, $\ubf_{k}$, is,
\begin{equation}
\ubf_{k} = \sum_{\forall \abf,  \mathcal{C(\abf)} = k} \ubf_\abf = \frac{1}{\Gamma} {N \choose k} \prod_{j=1}^k e^{-\beta f(j-1)} \\[10pt]
\end{equation} 
\noindent Where the normalization factor $\Gamma$ can also be simplified as: 
\begin{equation}
\Gamma = \sum_{k=0}^N {N \choose k} \prod_{j=1}^k e^{-\beta f(j-1)}
\end{equation}
\end{proof}
\newpage
%\section*{Supplementary References}
\bibliographystyle{unsrt}
\bibliography{bibliography.bib }
\end{document}